\documentclass{article}

\usepackage[utf8]{inputenc}
\usepackage[T1]{fontenc}
\usepackage[english]{babel}
\usepackage[left=3cm, right=3cm, bottom=3cm, top=3cm]{geometry}
\usepackage{fancyhdr}
\usepackage{graphicx}
\usepackage{amsmath}
\usepackage{amssymb}
\usepackage{mathrsfs}
\usepackage{eufrak}
\usepackage{qtree}
\usepackage{tipa}
\usepackage{indentfirst}
\usepackage[shortlabels]{enumitem}
\usepackage{stmaryrd}
\usepackage{qtree}
\usepackage{color}
\usepackage{ebproof}
\usepackage{listings}
\usepackage{hyperref}
\usepackage[disable,colorinlistoftodos,prependcaption]{todonotes}
\usepackage{datetime}
\newcommand{\mtodo}[1]{\todo[inline]{{#1}}}

\makeatletter
\newskip\@bigflushglue \@bigflushglue = -100pt plus 1fil
\def\bigcenter{\trivlist \bigcentering\item\relax}
\def\bigcentering{\let\\\@centercr\rightskip\@bigflushglue%
\leftskip\@bigflushglue
\parindent\z@\parfillskip\z@skip}
\def\endbigcenter{\endtrivlist}
\makeatother

\setcounter{tocdepth}{3}

\MakeRobust{\overrightarrow}

\definecolor{mygreen}{rgb}{0,0.6,0}
\definecolor{mygray}{rgb}{0.5,0.5,0.5}
\definecolor{mymauve}{rgb}{0.58,0,0.82}

\lstset{ %
  backgroundcolor=\color{white},      % choose the background color; you must add \usepackage{color} or \usepackage{xcolor}; should come as last argument
  basicstyle=\footnotesize,           % the size of the fonts that are used for the code
  breakatwhitespace=false,            % sets if automatic breaks should only happen at whitespace
  breaklines=true,                    % sets automatic line breaking
  captionpos=b,                       % sets the caption-position to bottom
  commentstyle=\color{cyan},          % comment style
  deletekeywords={...},               % if you want to delete keywords from the given language
  escapeinside={\%*}{*)},             % if you want to add LaTeX within your code
  extendedchars=true,                 % lets you use non-ASCII characters; for 8-bits encodings only, does not work with UTF-8
  frame=L,	                          % adds a frame around the code
  keepspaces=true,                    % keeps spaces in text, useful for keeping indentation of code (possibly needs columns=flexible)
  keywordstyle=\bf\color{green!40!black},      % keyword style
  language=Haskell,                       % the language of the code
  morekeywords={*,...},               % if you want to add more keywords to the set
  %numbers=left,                      % where to put the line-numbers; possible values are (none, left, right)
  %numbersep=5pt,                     % how far the line-numbers are from the code
  %numberstyle=\bf\tiny\color{blue},  % the style that is used for the line-numbers
  rulecolor=\color{black},            % if not set, the frame-color may be changed on line-breaks within not-black text (e.g. comments (green here))
  showspaces=false,                   % show spaces everywhere adding particular underscores; it overrides 'showstringspaces'
  showstringspaces=true,              % underline spaces within strings only
  showtabs=false,                     % show tabs within strings adding particular underscores
  stepnumber=1,                       % the step between two line-numbers. If it's 1, each line will be numbered
  stringstyle=\color{mymauve},        % string literal style
  tabsize=4,	                      % sets default tabsize to 2 spaces
  title=\lstname                      % show the filename of files included with \lstinputlisting; also try caption instead of title
}

\pagestyle{fancy}
\lhead{A deriving mecanism for Coq}
\rhead{Luc Chabassier}

\newcommand{\vc}[1]{\overrightarrow{#1}}
\newcommand{\mirrorh}[1]{\raisebox{\depth}{\scalebox{1}[-1]{#1}}}
\newcommand{\sss}{\subset}
\renewcommand{\ss}{\subseteq}
\newcommand{\rsss}{\supset}
\newcommand{\rss}{\supseteq}
\newcommand{\lbr}{\llbracket}
\newcommand{\rbr}{\rrbracket}
\newcommand{\sem}[1]{\lbr {#1} \rbr}
\newcommand{\msig}[1]{\langle {#1} \rangle}
\renewcommand{\L}{\mathscr{L}}
\newcommand{\C}{\mathscr{C}}
\newcommand{\V}{\mathscr{V}}
\newcommand{\K}{\mathfrak{K}}
\newcommand{\F}{\mathscr{F}}
\newcommand{\suc}{\mathbf{s}}
\newcommand{\con}{\mathbf{con}}
\newcommand{\args}{\mathbf{Args}}
\newcommand{\lf}{\mathbf{leaf}}
\newcommand{\fld}{\mathbf{Fold}}
\newcommand{\irule}[2]{\begin{array}{c} {#1} \\\hline {#2}\\ \end{array}}
\qtreecenterfalse

\title{A deriving mecanism for Coq\\
    {\normalsize Rapport de stage de L3}}
\author{Luc Chabassier\\
    Computer Science Departement\\
    École Normale Supérieure
}
\date{\today}

\begin{document}
\maketitle
\vfill
\tableofcontents
\vfill
Internship done with team MARELLE at INRIA Sophia-Antipolis from the
\formatdate{12}{6}{2017} to \formatdate{4}{8}{2017} under the direction of
Enrico Tassi.
\newpage


\section{Introduction}

Some programs are doubly generics in that they can be defined over almost any
datatype, because they follow a pattern. One such example is the equality
function, which implementation can be deduced from the datatype definition. These
programs are thus tedious to write since they are easy and repetitive, basically
just following the datatype definition.

\subsection{Haskell example}

Because of this, the Haskell standard impose the compilers to be able to \emph{derive}
typeclasses instances for the typeclasses \texttt{Eq}, \texttt{Ord}, \texttt{Enum},
\texttt{Bounded}, \texttt{Show} and \texttt{Read}. As can be seen in the
figure \ref{woDeriv}, the code gains in concision, readability and resilience
since there is no implementation to update when the datatype is changed.

\begin{figure}
    \begin{lstlisting}
    data LTerm = Var Integer | App LTerm LTerm | Abs LTerm

    instance Eq LTerm where
        Var i     == Var j     = i == j
        App x1 y1 == App x2 y2 = x1 == x2 && y1 == y2
        Abs x     == Abs y     = x == y
        _         == _         = false
    \end{lstlisting}

    \begin{lstlisting}
    data LTerm = Var Integer | App LTerm LTerm | Abs LTerm deriving (Eq)
    \end{lstlisting}
    \caption{With and without derivation for De Brujin $\lambda$-terms}
    \label{woDeriv}
\end{figure}

While this is all well and good, the programmer might want instances for other
classes to be derived. For example \texttt{map} (ie instances of \texttt{functor})
is a function that also follow a pattern. And indeed \emph{GHC} has extensions
to derive instances of \texttt{Functor}, but also \texttt{Foldable} and
\texttt{Traversable} for example. Despite this, the fact the programmer has to
rely to a non-standard extension of a compiler to derive is worrisome, and if he
wants to add derivation for other typeclasses he needs to patch the compiler.
This has led to many proposals and research to define a way to describe derivation
patterns directly in Haskell~\cite{genDerHaskell10}~\cite{reflDer97}~\cite{genDep},
even going all the way to an embedded language for meta programming~\cite{templateHaskell}.

\subsection{Going back to Coq}

Now that we've seen the benefits of a deriving mechanism, let's look at the Coq
situation : there is nothing, not even research on the subject. The closest
things that can be found is \emph{Coq in Coq}\cite{coqincoq}, which is more of
a formalisation of Coq's calculus of inductive construction in Coq than a
meta Coq programming attempt. Actually this is not completely true : Coq does
implicit derivation of an induction principle when a datatype is defined, but
it is quite weak (the generated principle is useless as soon as nested types
appear).

We could go the same way and implement other derivations directly in Coq, but
that would mean that adding new instances is complicated, and adding a
non-standard derivation mechanism would mean maintaining a patch over Coq. Using
the plugin interface seems better in that case, but it still requires good
knowledge of Coq internals, and is prone to be broken by Coq updates. Thankfully
another solution exists : a $\lambda$-prolog\cite{HiOrProg} interpreter has
been embedded into Coq, ELPI\cite{ELPI} and comes with a API to manipulate
Coq terms.

\subsection{ELPI : Embedded Lambda Prolog Interpreter}

Let's have a more detailed to this last option. $\lambda$-prolog brings us
backtracking (which won't really be used in the rest of the paper, but can be
used to easily implement types unification\cite{ELPItype}) and higher order unification
for free. The latter is particularly useful when extracting information from
Coq terms. On the other hand, the ELPI and its Coq API are both fairly recent,
and thus lack feedback, thus still having still some small bugs and more
importantly lacking debugging tools. There is as of now no way to track which generated
term caused a Coq error, for example.

Let's talk a bit more about the generated terms. The API exposes an abstract
syntax to manipulate them, and interact with Coq. This means the only thing we
can generate are constants (function, theorems but not types). The term themselves
are basically $\lambda$-calculus with a match construct, types annotations, a
fixpoint operator and a specific abstraction for types. See figure \ref{LambdaCoq}
for a more detailed definition.

\begin{figure}
    \begin{center}\begin{tabular}{cc}
        $\begin{array}{rcll}
            \Lambda & := & \Lambda\Lambda & \text{Application} \\
                    & |  & \lambda (\V : \Lambda). \Lambda & \text{Abstraction} \\
                    & |  & \forall (\V : \Lambda). \Lambda & \text{Product} \\
                    & |  & \text{fix } (\V : \Lambda)\ \Lambda & \text{Fixpoint} \\
                    & |  & \text{match } \V\ [\Lambda] & \text{Match} \\
                    & |  & \C & \\
        \end{array}$ & $\begin{array}{rcll}
            \C & := & \V & \text{Var} \\
               & |  & \text{const \emph{Str}} & \text{Constant} \\
               & |  & \text{indc \emph{Str}}  & \text{Constructor} \\
               & |  & \text{indt \emph{Str}}  & \text{Type} \\
        \end{array}$ \\
    \end{tabular}\end{center}
    \caption{Coq $\lambda$-calculus as seen from ELPI's API}
    \label{LambdaCoq}
\end{figure}

What is particularly interesting is that instead of using an abstract syntax tree,
what's used is an \emph{higher order abstract syntax} (HOAS). What that means is
that there is no \texttt{Var} constructor : when abstracting over a term, what
is given is a function from a term to a term, and thus the $\beta$-reduction
of the application of this term is simply the application of the function to
its argument. This makes it very easy to compute on this syntax, since all the
work is done by the language. Furthermore, thanks to how $\lambda$-prolog works,
it doesn't makes it that harder to analyse the syntax (while in other languages
functions are opaque containers).

\section{Deriving decidable equality instances}\label{deceq}

All the code is available at \url{https://github.com/LPCIC/coq-elpi}. The file
where the generation is done is \texttt{eq.elpi}
and the corresponding Coq file with the examples is \texttt{theories/test\_eq.v}.

This implementation was done without any formalisation on the types manipulated
nor forethought, resulting in a working but limited implementation.

\begin{figure}
    \begin{center}$\begin{array}{rcl}
        \K & := & * \\
           & |  & \text{indt \emph{Str}} \\
           & |  & \K\rightarrow\K \\
    \end{array}$\end{center}
    \caption{Kind definition}
    \label{kinds}
\end{figure}

\begin{figure}
    \begin{verbatim}
Inductive Nest (A : Type) :=
| NNil  : Nest A
| NCons : A -> Nest (A * A) -> Nest A.
    \end{verbatim} 

    \begin{verbatim}
Inductive Tree (A : Type) :=
| Leaf : A -> Tree A
| Node : list (Tree A) -> Tree A.
    \end{verbatim}
    \caption{\texttt{Nest} and \texttt{Tree} type definitions}
    \label{types}
\end{figure}

\newpage
\subsection{Types handled}

ELPI's API describes types with two information~:\begin{description}
    \item[kind] It's the type of the type, or more precisely a description of
        the parameters of the type. See figure \ref{kinds} for a formal
        definition of kinds. Notice how a parameter can be a term, this is
        for example used when encoding the length of a list inside its type.
        This specific case will \emph{not} be considered in this work. Parameters
        themselves can be one of the following~:\begin{description}
            \item[uniform] A uniform parameter is one which remain unchanged
                in the arguments of the constructors. For example, a list parameter
                is uniform since to build a list of $A$, you need a list of $A$.
            \item[non-uniform] A non-uniform parameter can changes in the arguments
                of the constructors. An example is the parameter of the
                \texttt{Nest} type defined in figure \ref{types}. But the parameter
                remain unchanged in the result of the constructors, as with
                uniform parameters. They won't be handled in this generation
                but will supported in section \ref{gender}.
            \item[index] Indexes are parameters which can change in the result
                of the constructors. They are mostly used to add information
                in the type~: for example the \texttt{vector} Coq type is basically
                a list with a \texttt{nat} index which record its size, and is
                thus increased by every \texttt{Cons} construction.
        \end{description}
    \item[constructors] Constructors can be seen has function which takes arguments
        and return an instance of the type, except they have no body. A type is
        recursive if one of its constructor takes an instance of itself as
        argument. The generation makes some assumptions on the types of the
        constructors~:\begin{itemize}
            \item their is no \emph{nested instance} of the type in the arguments
                of the constructor, the Type does not appear as parameter of
                other types, as seen in type
                \texttt{Tree} defined in figure \ref{types}. This case will be
                handled in section \ref{gender}.
            \item the arguments of the constructors are not dependants, meaning
                that the type of an argument cannot depend on the value of
                a previous argument.
        \end{itemize}
\end{description}

\subsection{Deriving an equality function}

The structure of an equality function is pretty easy to describe on a type of
kind $*$~: the two instances are destructed, and if they were built with
the same constructor the equality of the constructor arguments are compared,
otherwise the to instance are different and false can be returned.

Handling parameters is has easy : for each parameter, the equality function
takes as arguments an equality function for the parameters, and uses it to
compare their instances in the arguments of the constructors. The resulting
type for the equality for the list would thus be
$\forall (A : Type), (A \rightarrow A \rightarrow Bool) \rightarrow
list A \rightarrow list A \rightarrow Bool$.

This leaves open the question of how to compare others types. Let's say for
example that a constructor takes a \texttt{nat} as argument, the function needs
to know how to compare it. One possibility would be to add an argument to the
function which would be an equality function for \texttt{nat}. But this fall
short when the arguments are types with parameters~: what if the argument of
the constructor is a \texttt{list $A$} where $A$ is a parameter of the type
we're working on ?

One solution would be to use a database of equality functions for types, and
looking into it when a comparison function is needed, using the $\lambda$Prolog
unification to identify parameters of type and looking for their equality function
recursively. The algorithm is roughly :\begin{enumerate}
    \item need to compare \texttt{list $A$}, have comparison function $f_{list}$ of type
        $\forall (T : Type), (T \rightarrow T \rightarrow Bool) \rightarrow
        list T \rightarrow list T \rightarrow Bool$.
    \item identify $T$ with $A$, a function of type $A \rightarrow A \rightarrow Bool$
        is thus needed
    \item use the argument $f_A$ taken for parameter $A$
    \item return $f_{list}\ A\ f_A$
\end{enumerate}

It is quite easy to do thanks to the unification feature of $\lambda$Prolog,
but an outside of Coq database needs to be maintained and updated when a new
function is derived. And it is wasteful because Coq already has a mechanism
which can do it for us : \emph{canonical structures}.

\subsection{Canonical structures to the rescue}

Canonical Structures were introduced in Coq to provide ad-hoc polymorphism,
similarly to the typeclasses in Haskell. The idea is that instead of having
parameters which are of type \texttt{Type}, we create a record containing
a type, and additional structure for that type (in our example an equality
function over that type). It would give a signature of the form
$\forall (A : Eq), list (type A) \rightarrow list (type A) \rightarrow Bool$ for the comparison
on lists. The problem is still the same, one would say, since instead of
giving a type and an equality function, we give a record with a type and its
equality function inside.

That's were canonical structure play a role. If we leave a \emph{hole} in the
type argument, Coq unification algorithm has to resolve $T = type\ ?A$ with
$A : Eq$, when applying to a \texttt{list $T$}. It can't do that on its own,
but if we've already defined an $Eq$ for $T$, we can add an hint to the typer
telling it to use that $Eq$ for $T$ : this is the canonical structure mechanism.
Now we can call the function on \texttt{list $T$} and Coq will automatically
know which comparison function to use.

But the mechanism is even more powerful, since now we can add an $Eq$ instance
for \texttt{list $T$} when $T$ has an instance as $Eq$ associated. In other
words, instances of canonical structures can have dependencies, which are
also automatically solved by Coq. For a more in-depth description of canonical
structures, please see \cite{CSCoq13}.

Now our problem resolves itself : when comparing two instances of a type $type E$,
we have the equality function given inside $E$, since Coq has done all the
unification work.

We're left with the definition of the canonical structure we want to work with.
A structure packing a type and an equality function on would be sufficient
if we were to only derive the function, as we would do in Haskell. But in Coq
we've got more power : our function only has value if it is correct (we don't
want a user to provide a constant function for example), so we also want to
pack a proof of the correctness of our function with it, something that
says :
\[\forall (x\ y : T), f_T\ x\ y = true \leftrightarrow x = y\]

So the record will pack a type, an equality function on it and the proof of
correctness of the function. From a computational point of view we've got what we
want but it can be interpreted another way : saying that the equality on a
type is equivalent to a boolean function means that equality is decidable for
this type. With that in mind, it turns out \emph{MathComp} already has a
canonical structure for this called \texttt{EqType}. Defining another one was
still preferred to avoid adding \emph{MathComp} to the dependencies, but it
could be ported with minimal effort. Furthermore Coq already has a
\emph{decide equality} tactic, which works in the same conditions as our
generation, and basically does the same, but without using canonical structures.

\subsection{Deriving the proof}

The structure of the proof is basically the same as the equality function, but
instead of doing the fixpoint manually the generated induction principle was
used. The difficulty here was that the API only allows us to write terms, and
not use tactics, thus the whole proof term needed to be hand-generated, and
while it was easy it still took a long time.

The choice of using the induction principle is what restricted us to uniform
parameters, since the induction principle for types with non-uniform parameters
in not powerful enough for this use case (it doesn't mingle well with canonical
structure usage).

\section{Toward a generic deriving mechanism}\label{gender}

In the previous section, we described how both the equality function and its
correctness proof were computed. They both followed a similar pattern, and
yet their codebase are completely disjunct. This calls for a common API
simplifying the description of derivation schemes, or even better a generic
derivation description mechanism.

To do that, we first need to precisely describe which types we're working with.
Despite the lack of literature on the subject in Coq, Haskell has plenty to
choose from. The formalism we will describe now has been heavily inspired by
the article \cite{genDep}.

\subsection{Polynomial type formalism}

\subsubsection{Type definition}

Our formalism will allow us to describe nested, non-uniform parametric
polynomial types. From now on the term product will be used to describe a pair,
since pairs are made by taking the product of two types, and not the meaning
it has in Coq, describing dependent types.

First we need to define kinds. The definition in figure \ref{kinds} is the
one we will use, without the possibility to have terms on types. A type
first needs to have its kind fixed, and since we want to handle mutually recursive
(nested) datatype, we define $Sig$ to be a list of kinds, the list of the kinds
of the type we're defining~:
\[\begin{array}{cc}
    \irule{}{\epsilon : Sig} & \irule{K : \K\quad S : Sig}{K;S : Sig}
\end{array}\]

Let's notice that kinds can be seen as $Sig$. Indeed, a kind is a list of kinds
(its parameters) and it returns an object of kind  $*$. All the information is
thus the parameters, in other world a Sig. Thus we introduce the symbol $\msig{.}$
to denote this conversion~:
\[\irule{K : \K}{\msig{K} : Sig}\]

We now define a way to refer to the nth element of a $Sig$ in a type safe
manner~:
\[\begin{array}{cccc}
      \irule{S : Sig\quad K : \K}{Var\ S\ K : Type}
    & \text{where}
    & \irule{S : Sig\quad K : \K}{0 : Var\ (K;S)\ K}
    & \irule{v : Var\ S\ K\quad J : \K}{\suc v : Var\ (J;S)\ K} \\
\end{array}\]

We must now define the polynomial representation of a type. A polynomial
definition is parametrized by $\Delta$ the list of kinds of the types we're
defining, $\Lambda$ the list of its parameters and $K$ the kind of the expression.
$V$ is used to refer to a parameter and $D$ to one of the type we're defining.
Here we differ from the original article in that instead of generally defining
polynomial types, we define them in sum-of-products forms, leaving us the
opportunity to add metadata for the user in the implementation.

\[  \begin{array}{c}
        \irule{\Delta, \Lambda : Sig\quad K : \K}{Tyl\ \Delta\ \Lambda\ K : Type} \\
        \irule{\Delta, \Lambda : Sig}{Typ\ \Delta\ \Lambda : Type} \\
        \irule{\Delta, \Lambda : Sig}{Ty\ \Delta\ \Lambda : Type} \\
    \end{array}
    \text{ where }
    \begin{array}{cc}
        \irule{v : Var\ \Delta\ K}{Dv : Tyl\ \Delta\ \Lambda\ K}
            & \irule{v : Var\ \Delta\ K}{Vv : Tyl\ \Delta\ \Lambda\ K} \\
        \multicolumn{2}{c}{\irule{F : Tyl\ \Delta\ \Lambda\ (J\rightarrow K)\quad
                X : Tyl\ \Delta\ \Lambda\ J}
            {F\cdot X : Tyl\ \Delta\ \Lambda\ K}} \\
        \irule{}{1 : Typ\ \Delta\ \Lambda}
            & \irule{S, T : Typ\ \Delta\ \Lambda}{S \times T : Typ\ \Delta\ \Lambda} \\
        \irule{}{0 : Ty\ \Delta\ \Lambda}
            & \irule{S, T : Ty\ \Delta\ \Lambda}{S + T : Ty\ \Delta\ \Lambda} \\
        \irule{P : Typ\ \Delta\ \Lambda}{\con P : Ty\ \Delta\ \Lambda}
            & \irule{L : Tyl\ \Delta\ \Lambda\ *}{\lf L : Typ\ \Delta\ \Lambda} \\
    \end{array}
\]

The type of the definition of a type of kind $K:\K$ is thus $Ty\ \Delta\ \msig{K}$.

We can now assign a meaning to a signature $\Delta$ as the function which takes
a variable selecting one of the kinds in it and returns the polynomial type
definition for it~:

\[\irule{\Delta : Sig}{\sem{\Delta} : \forall (K : \K), Var\ \Delta\ K\rightarrow
Ty\ \Delta\ \msig{K}} \]

In the rest of the paper we will assume $K$ to be implicit, since it can be
deduced from the value of the $Var$.

\subsubsection{Populating those types}

After giving a way to define the types, we need to define constructors for
them. We will work with an arbitrary definition $\delta = \sem{\Delta}$. We
thus want to give an interpretation to a type definition~:

\[\begin{array}{ccc}
    \irule{T : Ty\ \Delta\ \epsilon}{\sem{T}_\delta : Type}
    & \irule{T : Typ\ \Delta\ \epsilon}{\sem{T}_\delta : Type}
    & \irule{T : Tyl\ \Delta\ \epsilon\ *}{\sem{T}_\delta : Type}
    \\
\end{array}\]

Notice how the types are all fully applied. We need to introduce a way to work
with a list of parameters. We represent it as function space, with type
$\args$~:

\[\irule{\Delta : Sig\quad K : \K}{\args\ \Delta\ K : Type}
    \text{ where }
    \args\ \Delta\ K = \forall (J : \K), Var\ \msig{K} J\rightarrow Tyl\ \Delta\ \epsilon\ J
\]

To make it easier to work with it, we introduce two new operators : application
and substitution by a list of parameters (their semantic is obvious, so we wont
detail it).
\[\begin{array}{cc}
      \irule{X : Tyl\ \Delta\ \epsilon\ K\quad \vc{Y} : \args\ \Delta\ K}
        {X@\vc{Y} : Tyl\ \Delta\ \epsilon\ *}\text{(app)}
    & \irule{X : Tyl\ \Delta\ \msig{J}\ K\quad \vc{Y} : \args\ \Delta\ J}
        {X[\vc{Y}] : Tyl\ \Delta\ \epsilon\ K}\text{(subst)} \\
\end{array}\]

We can now express the constructors.

\[\begin{array}{cccc}
      \irule{s : \sem{S}_\delta}{\text{inl}\ s : \sem{S + T}_\delta}
    & \irule{t : \sem{T}_\delta}{\text{inr}\ t : \sem{S + T}_\delta}
    & \irule{}{\text{unit} : \sem{1}_\delta}
    & \irule{s : \sem{S}_\delta\quad t : \sem{T}_\delta}{\text{pair}\ s\ t : \sem{S\times T}_\delta}
    \\
      \irule{t : \sem{(\delta\ v)[\vc{X}]}_\delta}{\text{typ}_v\ t : \sem{D v@\vc{X}}_\delta}
    & \irule{S : Typ\ \Delta\ \Lambda\quad t : \sem{S}_\delta}{\text{con}\ t : \sem{\con S}}
    & \irule{S : Tyl\ \Delta\ \Lambda\ *\quad t : \sem{S}_\delta}{\text{leaf}\ t : \sem{\lf S}}
    &
    \\
\end{array}\]

This representation allows us to describe inductive type, eventually nested,
with both uniform and non-uniform parameters, but not indexes.

\subsubsection{Interpretation and example}

What we must notice here is that a regular Coq type of kind $K$ has type
$Tyl\ \Delta\ \msig{K}\ *$, which is constructed using $\text{typ}_v$ where
$v$ is the index of the definition of the type in $\Delta$. Thus we have a
type in the form of sum of products of basic Coq types. The con and leaf constructor
are used to add metadata, like which constructor are we using and which argument
of the constructor are we in. This is not relevant in our theory but may be
useful when writing concrete generators.

Let's now look at an example to understand a bit better the notation. We'll use
the \texttt{Tree} type from figure \ref{types}. Since it uses a \texttt{list}
we'll assume it is defined at the same time. Both \texttt{list} and \texttt{Tree}
have kind $*\rightarrow *$, so the signature we'll use is
$\Delta = *\rightarrow *; *\rightarrow *; \epsilon$. We decide \texttt{list} is the first
one and \texttt{Tree} the second one. We can now give the interpretation of our
signature (ie the definition of the two types)~:

\[\sem{\Delta} = \left\{\begin{array}{rcl}
    0 & \mapsto & \con_{\text{nil}}\ \lf_0\ 1 + \con_{\text{cons}}\ (\lf_0\ V0 \times \lf_1\ (D0\cdot V0)) \\
    \suc 0 & \mapsto & \con_{\text{Leaf}}\ \lf_0\ V0
        + \con_{\text{Node}}\ \lf_0\ (D0\cdot (D\suc 0\cdot V0)) \\
\end{array}\right.\]

Metadata annotations were added to $\con$ and $\lf$ to make it easier to link
the definition here with the one we would use in Coq.

Let's have a look at what would give an instance, for example a tree of \texttt{nat}
with one leaf (we will use \texttt{nat} as a black box, and not give its
definition)~:
\begin{verbatim}
    Node (cons (Leaf 42) nil)
\end{verbatim}

The corresponding representation would thus be~:
\begin{align*} \text{typ}_{\suc 0} (\text{inr}\ \text{con}_\text{Node}\ \text{leaf}_0
    (&\text{typ}_0 \text{inr}\ \text{con}_\text{cons}\ (\text{pair} \\
        &(\text{leaf}_0\ \text{typ}_{\suc 0}\ \text{inl}\ \text{con}_\text{Leaf}\ \text{leaf}_0 42) \\
        &(\text{leaf}_1\ \text{typ}_0\ \text{inl}\ \text{con}_\text{nil}\ \text{unit})
    ))
)
\end{align*}

\subsection{First approach : a generic fold}

Having found a nice representation of types, we must now find a way to describe
computations over it. Our aim is not to be able to describe every possible
computation, but to find a compromise between a clean API and a good
expressiveness.

\subsubsection{Theory}

First we need to define the type of our operation. We call the type of our
operation $\fld$. It will need a few parameters~:\begin{itemize}
    \item $K$ the kind of the type we're working on.
    \item $X$ the type we're working on, of kind $K$.
    \item $\mathbf{Y}$ a $n$-ary vector of types, all of kind $K$.
    \item $\Phi$ is the user provided function which allows to control the
        resulting type. It takes a term as argument as we may want to generate
        proofs, which may require dependent types.
\end{itemize}

More precisely, we have~:

\[\begin{array}{cc}
    \irule{X : Ty\ \Delta\ \epsilon\quad \mathbf{Y} : (Ty\ \Delta\ \epsilon)^n\quad a : X}
        {\Phi\ a\ X\ \mathbf{Y} : Type}
    & \irule{X : Typ\ \Delta\ \epsilon\quad \mathbf{Y} : (Typ\ \Delta\ \epsilon)^n\quad a : X}
        {\Phi\ a\ X\ \mathbf{Y} : Type}
    \\
    \multicolumn{2}{c}{
        \irule{X : Tyl\ \Delta\ \epsilon\ *\quad \mathbf{Y} : (Tyl\ \Delta\ \epsilon\ *)^n\quad a : X}
        {\Phi\ a\ X\ \mathbf{Y} : Type}}
    \\
\end{array}\]

$\fld$ is defined recursively on the of the type handled (any operator with a
$^n$ means the operator operator point to point on $n$-ary vectors) as follow~: 
\begin{align*}
    \fld_*\ \Phi\ S\ \mathbf{T} &= \forall (a : \sem{S}_\delta),\Phi\ a\ S\ \mathbf{T} \\
    \fld_{J\rightarrow K}\ \Phi\ F\ \mathbf{G} &= \forall (X : Tyl\ \Delta\ \epsilon\ J)
        \ (\mathbf{Y} : (Tyl\ \Delta\ \epsilon\ J)^n),
        \fld_J\ \Phi\ X\ \mathbf{Y} \rightarrow\fld_K\ \Phi\ (F\cdot X) (\mathbf{G}\cdot^n\mathbf{Y})
\end{align*}

Let's now identify which parameters an instance of $\fld_K\ \Phi\ S\ \mathbf{T}$
must have. The idea is that their is a parameter by constructor, describing
how to build the resulting value, which $\text{fold}_X$ can combine to create
the whole computation.

Here are the typing rules we expect~:
\[\begin{array}{cc}
    \multicolumn{2}{c}{\irule{\phi : \Phi\ a\ ((\delta v)[\vc{W}])\ ((\delta v)^n[\vc{\mathbf{Z}}]^n)}
        {doType\ v\ \phi : \Phi\ (\text{typ}_v\ a)\ (D v @ \vc{W})\ ((D v)^n@^n\vc{\mathbf{Z}})}} \\
    \irule{\phi_1 : \Phi\ a\ S_1\ \mathbf{T_1}}
            {doInl\ \phi_1 : \Phi\ (\text{inl}\ a)\ (S_1 + S_2)\ (\mathbf{T_1} +^n \mathbf{T_2})}
        & \irule{\phi_2 : \Phi\ a\ S_2\ \mathbf{T}_2}
            {doInr\ \phi_2 : \Phi\ (\text{inr}\ a)\ (S_1 + S_2)\ (\mathbf{T_1} +^n \mathbf{T_2})} \\
    \irule{}{doUnit : \Phi\ \text{unit}\ 1\ 1^n}
        & \irule{\phi_1 : \Phi\ a_1\ S_1\ T_1\quad\phi_2 : \Phi\ a_2\ S_2\ T_2}
            {doPair\ \phi_1\ \phi_2 : \Phi\ (\text{pair}\ a_1\ a_2)\
                (S_1\times S_2)\ (\mathbf{T_1}\times^n\mathbf{T_2}} \\
    \irule{\phi : \Phi\ a\ S\ \mathbf{T}}{doCon\ \phi : \Phi\ (\text{con}\ a)\ (\con S)\ (\con^n\mathbf{T})}
        & \irule{\phi : \Phi\ a\ S\ \mathbf{T}}
            {doLeaf\ \phi : \Phi\ (\text{leaf}\ a)\ (\lf S)\ (\lf^n\mathbf{T})} \\
\end{array}\]

And with this, we want~:

\[ \irule{X : Tyl\ \Delta\ \epsilon\ K}
    {\text{fold}_X\ doType\ doInl\ doInr\ doUnit\ doPair\ doCon\ doLeaf : \fld_K\ \Phi\ X\ X^n} \]

\subsubsection{Implementation}

Let's get more concrete now. We want to find a system were the user gives $\Phi$, $n$,
the $do^*$, and generates the appropriate fold. Of course our API is not as
simple as above, since we add a lot of metadata to make the writing of terms
easier.

The point is to write a set of $\lambda$Prolog relations, one for each arguments
of fold, which starts with the name of the generator. That way multiple definitions
can be accumulated on the same execution, and they can be referred to. This is
what the \texttt{Name} argument will be. Other arguments present in every relation
are \texttt{Type} the Coq type we are
generating for, and \texttt{Tp}, the type of the term we are actually
handling. In other words, when handling the left part of a sum that belongs to
the definition of a type, \texttt{Type} will be that type and \texttt{Tp} will
be the type of the left of the sum.

Let's now give the corresponding API~:\begin{itemize}
    \item \texttt{params Name N}~: set the $n$ of the generation
    \item \texttt{phi Name S T A Out}~: \texttt{S}, \texttt{T} and \texttt{A}
        are the arguments of $\Phi$ (meaning that \texttt{T} is a list of types
        of length $n$). \texttt{Out} is, as for the other rules, the resulting
        term.
    \item \texttt{doType Name Type Tp A Phi Out}~: \texttt{A} is the term
        of type \texttt{Type}, and \texttt{Phi} is the result of the recursive
        call.
    \item \texttt{doCon Name Type Tp C A Phi Out}~: the only difference is
        the presence of \texttt{C}, which is metadata telling which Coq
        constructor would be used here.
    \item \texttt{doLeaf Name Type Tp Phi Out}~: $doLeaf$ does not have metadata
        for now, and in most case the user would want it to be the identity,
        but it could provide the id of the argument of the constructor we're
        considering.
    \item \texttt{doInl Name Type Tp A Phi Out} and \texttt{doInr Name Type Tp A Phi Out}
        are pretty straightforward.
    \item \texttt{doUnit Name Type Out}~: their is no \texttt{Tp} since it would
        always be the \emph{unit} type.
    \item \texttt{doPair Name Type Tp A Phi1 Phi2 Out}~: there is only one \texttt{A}
        since it is the pair, and not its deconstruction.
\end{itemize}

When generating the term the first thing to do is to abstract over the parameters.
This is done by using following the definition of $\fld$, by induction over the
kind of the considered type.

After that, since we want to generate a term of type $\forall (a : T), \Phi\ a\ S\ \mathbf{T}$,
we need to abstract over $T$ to get the element $a$. Once this is done, we must
convert this element to a representation of sum of products, but since we want
to do it at compile time, we need an instance of $T$, and not an abstract element.
Thus we match on $a$ to get all the possible instances, and in every branch we
apply the $do^*$ following the type of the element as described in the rules in
the previous section, nesting the resulting terms.

The only thing that isn't described in the previous section is how the recursive
call is done. Indeed, when reaching a leaf of type $T'$, we need to be able to
compute on it. $T'$ is necessarily of kind $*$, and is the application of a
type $Tp$ of kind $K$ to a list of parameters $L$. What we do is find a fold
for type $Tp$ (ie of type $\fld_K\ \Phi\ Tp\ Tp^n$). If we have one (either
taken as argument or because it is the type we're recursing on), we use it.
If we don't, the generator calls itself recursively to generate one for this
type, and inline it. Since we may not always want to inline (for example when
generating an equality for a type which uses \texttt{nat}, we may want to
use an already generated equality function), we need to add a mechanism for
the user to supply the function for some type. This is done by adding a
\texttt{doConst Name Type Tp Out} relation to the API where \texttt{Out} must
be a term of type $\fld_K\ \Phi\ Tp\ Tp^n$.

We then need to apply it to the right arguments. This is easy since we know the
list of parameters (and we also know the additional parameters, for $\mathbf{T}$).
Once they are applied, we need the appropriate fold to give as argument. This is
obtained by recursively applying our research procedure for the new type. This
procedure always terminates since the kind of the considered type decreases at
each recursive call.

\begin{figure}
    \begin{verbatim}
Fixpoint tree_equal (A : Type) (f : A -> A -> bool) (a b : Tree A) :=
match a, b with
| Leaf _ x, Leaf _ y => f x y
| Node _ x, Node _ y => (fix list_equal (B : Type) (g : B -> B -> bool)
                                        (c d : list B) :=
  match c, d with
  | nil, nil => true
  | cons x lx, cons y ly => andb (g x y) (list_equal B g lx ly)
  | _, _ => false
  end) (Tree A) (tree_equal A f) x y
| _, _ => false
end.
    \end{verbatim}
    \caption{Nested recursion failing in Coq}
    \label{coqindfail}
\end{figure}

\begin{figure}
    \begin{verbatim}
Fixpoint tree_equal (A : Type) (f : A -> A -> bool) (a b : Tree A) :=
match a, b with
| Leaf _ x, Leaf _ y => f x y
| Node _ x, Node _ y => (fun (B : Type) (g : B -> B -> bool) =>
                             fix list_equal (c d : list B) :=
  match c, d with
  | nil, nil => true
  | cons x lx, cons y ly => andb (g x y) (list_equal lx ly)
  | _, _ => false
  end) (Tree A) (tree_equal A f) x y
| _, _ => false
end.
    \end{verbatim}
    \caption{Nested recursion working in Coq}
    \label{coqind}
\end{figure}

While this algorithm would work out of the box with Haskell, extra care must be
taken with Coq to ensure that Coq can infer the decreasing argument. Indeed, if
we look at the example figure \ref{coqindfail}, Coq fails to infer the decreasing
argument of \texttt{Tree\_equal}, while the example in figure \ref{coqind}
works. This is because Coq cannot $\beta$-reduce fixpoints, and thus only
see the recursive function as partially applied. On reaction would be to move
the abstraction over the parameters outside of the fixpoint. But this situation
would fail when a parameter is non-uniform, since it changes in the recursive
call it must be inside the fixpoint. The solution is to put the uniform
parameters outside the fixpoint and the non-uniform inside, but then two
problems arise~:\begin{itemize}
    \item If a type has a non-uniform parameter, then a uniform one, the order
        in which we abstract over them would be swapped when compared with
        what $\fld$ would give, making the generation less predictable (and
        more complicated). We're saved here by Coq limitations~: instead of
        tracking which argument is uniform and which one is non-uniform, it stores
        the number of consecutive uniform parameters of the type, and consider
        all the following parameters non-uniform. This is exactly what we want,
        since we can move the first parameters outside the fixpoint and keeping
        the non-uniform one inside.
    \item Since a part of the abstraction is moved outside of the fixpoint,
        the function obtain when looking for a fold for the type is not something
        of type $\fld$ but is already partially applied. This could be fixed
        by adding a special case to the procedure but it is much easier to
        abstract again over the recursive function, ignoring the first
        parameters. That way the recursive function is also a $\fld$.
\end{itemize}

One could remark that this doesn't change anything to the problem if the type
we're defining is a non-uniform parameters of another type, since the recursive
function would still need to be given to a fixpoint. Once again we're saved by
Coq constraints~: a type such as this fails the posivity constraint, and thus
can't be defined in Coq. It remains unclear wether this limitation is necessary
for the consistency of the theory, or is just a by-product of an unsound
positivity analysis (meaning it could be removed in following Coq versions).

At this point our generator works, but isn't expressive at all. Indeed, there
is no way to manipulates sum and products in the terms given to the API. Indeed,
to make a description of an equality function, one needs to manipulate a
second argument of the same type as the first one, or for a map function,
it is necessary to rebuild the type. To make that possible, the term where
extended with other constructions to manipulate those terms. Since Coq doesn't
about those, the generated term is first $\beta$-reduced (with special reduction
rules for the new terms) and simplified. Hopefully all the special terms
disappear during that step. If they don't, the generation fails.

The new constructions and their rule are given below (the syntax used here is not valid
$\lambda$Prolog syntax, it is used to give the intuition for the use of the
new terms). In the last rule, $x'$ designate the representation in terms of
sum of products of any instance of $x$.

\[\begin{array}{cc}
      \mathtt{match\_sum}\ (\mathtt{inl}\ x)\ f\ g\rightarrow f\ x
    & \mathtt{match\_sum}\ (\mathtt{inr}\ y)\ f\ g\rightarrow g\ y
    \\
      \mathtt{uncon}\ (\mathtt{con}\ c\ x)\ f \rightarrow f\ x
    & \mathtt{match\_prod}\ (\mathtt{prod}\ x\ y)\ f \rightarrow f\ x\ y
    \\
    \multicolumn{2}{c}{
        \mathtt{abstract\_type}\ T\ x\ f\rightarrow \mathtt{match}\ x\ \mathtt{with}\ [f\ x']
    } \\
\end{array}\]

\subsubsection{Results and limitations}

In the files \texttt{geneq.elpi} and \texttt{genmap.elpi} generators for an
equality function and a map function can be found. They are very easy to write
once the $\Phi$ has been figured. In the case of the equality function, we
have $\Phi_{\text{eq}}\ a\ S\ [] = S \rightarrow \text{bool}$~: the
equality partially applied with $a$, and not other parameters (ie $n = 0$). For
the map we set $n = 1$, and use the element in the list of parameters for the
result~: $\Phi_{\text{map}}\ a\ S\ [T] = T$. To convince ourselves this is the
right type, let's do it on a type of kind $*\rightarrow *$~:
\begin{align*}
    \fld_{*\rightarrow *}\ \Phi\ T\ [T]
        &= \forall (X\ Y : Tyl\ \Delta\ \epsilon\ *), \fld_*\ \Phi\ X\ [Y]
            \rightarrow \fld_*\ \Phi\ (T\cdot X)\ [T\cdot Y] \\
        &= \forall (X\ Y : Type), (\forall (a : X), \Phi\ a\ X\ [Y])
            \rightarrow (\forall (a : (T\cdot X)), \Phi\ a\ (T\cdot X)\ [T\cdot Y]) \\
        &= \forall (X\ Y : Type), (X \rightarrow Y) \rightarrow ((T\cdot X)\rightarrow(T\cdot Y)) \\
\end{align*}

This is indeed the type of a map we're used to see. It seems powerful enough for
functions, but what about proofs~? Well this were things get ugly. First of
all writing the $\Phi$ for a proof is complicated. Indeed, when stating a specific
equality function is correct, we state it for the whole type, and we have no
way to refer to specific subpart of the proof. One way to circumvent this problem
would be to generate a pair (or any record) of a function and its proof.

But this is only the tip of the iceberg. Inside the equality proof, there is
a need for projector function (ie projecting on the $n$-th argument of the $C$
constructor of the type). With the right metadata, it could be possible to create
them with helpers generators, and the proof would have to happen in the
$doCon$ and not $doIn^*$.

Yet it would still not work as often in a proof lemmas and external theorems
are used, and take as argument propositions, which are written with specific
terms in them, but which won't be simplified. Even if we complemented the
$\beta$-reduction by $\delta$-reduction (replacing named constants by their
body), some terms still wouldn't be possible to eliminate.

So even if it may have worked for some proofs, it would be awfully slow, and still
not work for every structural proof. What's more, it is very difficult to
properly define which proofs could be generated and which couldn't. This call
for another approach.

\subsection{Second approach : using Coq expressivity}

The intuition for this approach is that Coq is a very expressive language, so
we want to completely lift the generation in the Coq language. The inspiration
in Haskell Generic library~: we have a Coq type used to describe other types,
the user write functions for this type, and an ELPI program generates the
necessary conversion function to and from this type, along with some proofs
of correction.

More precisely, the idea is to have a type $A$, and for any polynomial type $T$
generate $f_T : T \rightarrow A$, $g_T : A \rightarrow\mathtt{option}\ T$, and
a proof of $\forall (x : T), g_T\ (f_T\ x) = \mathtt{Some}\ x$.

But defining this type $A$ in a manner which can be computed upon turned out
to be very complicated, more precisely in the handling of fixpoints. Haskell
circumvented this issue by making it a typeclass and heavily relying on the
recursive nature of the language.

Furthermore, the Haskell library only handles types of kind $*$, and trying
to generalize it might be done in Coq using dependants type, but defining any
kind of function on it become much more complicated.

\subsection{Reconciliation of the two approaches}

The second approach sound like it hit a wall, but its problems can be narrowed to
the two mentioned above~: \begin{itemize}
    \item It only works well with types of kind $*$. But the first approach gave
        us a way to define ourselves recursively on the kind until we reached
        the kind $*$, so maybe not all is lost.
    \item If Coq itself is not able to choose which function to apply when
        recursing, a specific $\lambda$Prolog generator can do~: it was what
        we were doing in the first approach.
\end{itemize}

So the idea is~: we define in Coq the sum, products, unit\ldots of types.
Then for each type $T$ we generate another isomorphic type $rep_T$, expressed
in terms of sum of products of standard Coq types. $T$ can appear inside,
but only with the condition that when appears in structurally smaller position
(without this condition, we could set $rep_T = T$, which wouldn't give us
any information). We also generate $f_T : T \rightarrow rep_T$, $g_T : rep_T\rightarrow T$
and proofs of $f . g = id_{rep_T}$ and $g . f = id_T$.

Then the user write a generator using the same API as what was described in the
first approach, but instead of using the specific terms, we can write standard
Coq terms, since now sums and everything are defined in Coq, replacing
\texttt{abstract\_type} by $f_T$. The simplification step isn't even necessary
anymore since $g_T$ is exposed to the user.

Which raises the question~: why not have do so since the beginning. Well there
is one thing that becomes much more complicated. In the first approach, when it
worked, there was no trace of conversion to another representation of the type
in the generated term, since it would have been simplified. While this is not
an issue for terms of proof, it adds complexity to the generated functions and
makes it more difficult for the user to make its own proof about them.

If the user uses the conversion function as an argument of its program $\phi$,
we get something of the form~:
\[\phi\ \left(\begin{array}{l}
    \mathtt{match}\ x\ \mathtt{with}\\
    | C_1 \ldots \Rightarrow t_1 \\
    \vdots \\
    | C_i \ldots \Rightarrow t_i \\
\end{array}\right)\]

This can't be simplified. If we look at how \texttt{abstract\_type} worked,
it gave us something of the form~:
\[\begin{array}{l}
    \mathtt{match}\ x\ \mathtt{with}\\
    | C_1 \ldots \Rightarrow \phi\ t_1 \\
    \vdots \\
    | C_i \ldots \Rightarrow \phi\ t_i \\
\end{array}\]

We can now simplify because $t_i$ is the representation of an instance in terms
of sums of products, and $\phi$ can now be applied. This means that to get a
simplification equivalent to the first approach, it would be necessary to add
this rewriting rule to raise the matches above the functions, and a proof that
it does not change the function. Due to lack of time this part has not been
implemented.

\section{Conclusion}

ELPI has shown to be a powerful way to write Coq plugins, but a complicated
one. Generating proof terms is a complicated task, albeit useful, and raised
the need for a higher-level API for our mechanism. Despite some troubles,
we found an API which seems promising, and further work on this direction
could prove useful. As for now the software will be made available as an Opam
package to see if there is interest from the community.

A few remarks though~: even if the last approach were to prove as good as it seems,
it would still require to write the proof terms by hand, which is tedious to
say the least. A better way to write proof is necessary, but maybe the solution
comes from improving ELPI Coq API.

% References
\newpage
\addcontentsline{toc}{section}{References}
\bibliography{sources}
\bibliographystyle{plain}

\end{document}

